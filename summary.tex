\RequirePackage[l2tabu, orthodox]{nag}
\documentclass[version=last, pagesize, twoside=off, bibliography=totoc, DIV=calc, fontsize=12pt, a4paper, french, english]{scrartcl}
\input{preamble/packages}
\input{preamble/math_basics}
\input{preamble/math_mine}
\input{preamble/redac}
\usepackage[normalem]{ulem}
%\input{preamble/draw}
%\input{preamble/acronyms}
\addtokomafont{labelinglabel}{\sffamily\bfseries}
\DeclareMathAlphabet{\mathup}{OT1}{\familydefault}{m}{n}

%I find these settings useful in draft mode. Should be removed for final versions.
	%Which line breaks are chosen: accept worse lines, therefore reducing risk of overfull lines. Default = 200.
		\tolerance=2000
	%Accept overfull hbox up to...
		\hfuzz=2cm
	%Reduces verbosity about the bad line breaks.
		\hbadness 5000
	%Reduces verbosity about the underful vboxes.
		\vbadness=1300

\begin{document}
\title{Theories of deliberated choice} LE TITRE ME SEMBLE INSUFFISAMMENT INFORMATIF,  ET POTENTIELLEMENT TROMPEUR : ON PEUT S'ATTENDRE A UNE SYNTHESE DE THEORIES EXISTANTES
\author{Olivier Cailloux}
\affil{Université Paris-Dauphine, PSL Research University, CNRS, LAMSADE, 75016 PARIS, FRANCE\\
  \href{mailto:olivier.cailloux@dauphine.fr}{olivier.cailloux@dauphine.fr}
}
\makeatletter
\hypersetup{
  pdfsubject={Epistemology},
  pdfkeywords={Decision aiding, Decision making, Argumentation} CONCERNANT "DECISION AIDING", NOUS EN AVONS DISCUTE A PLUSIEURS REPRISES, ET C'EST TOI-MEME QUI M'A MIS SOUS LES YEUX QU'A PART ALEXIS ET QUELQUES UNES DE SES CONNAISSANCES, PERSONNE N'UTILISAIT CE SYNTAGME. DEPUIS, J'UTILISE SYSTEMATIQUEMENT "DECISION SUPPORT", QUI ME PARAIT MOINS SECTAIRE
}
\makeatother
\maketitle

\section{Introduction}
Observing subjects perform acts of choice is the basis of much work in econometrics. JE DIRAIS "STUDYING", CAR ILS NE FONT PAS QU'OBSERBER, ET LE FONT PEUT-ETRE INSUFISAMMENT. PAR AILLEURS, POURQUOI "ECONOMETRICS" PLUTOT QU' ECONOMICS TOUT SIMPELMENT ?
Most of these approaches consider only acts of choice that are performed spontaneously, QUID DES EXPERIENCES : PEUT-ON DIRE QUE LE COBAYE D'UNE EXPERIENCE FAIT QUELQUE CHOSE DE "SPONTANNE" ? such as when studying willingness to pay or when studying reactions of individuals to varying frames as in the celebrated work of Kahneman and Tversky \citep{bell_descriptive_1988, kahneman_thinking_2012}. PLUTOT QUE DE LA REACTION A DES FRAMES, IL ME SEMBLE PLUTOT QUE C'EST L'INFLUENCE DE FRAMES SUR DES COMPORTEMENTS

On the other hand, since at least the seminal work of \citet{fishkin_when_2011}, another sort of judgment involving subjective desirability is emerging as also worth studying. LA PHRASE PRECEDENTE NE ME SEMBLE PAS BIEN COMPREHENSIBLE. PEUX-TU REFORMULER "INVOLVING SUBJECTIVE DESIRABILITY" ? C'EST SURTOUT LE "INVOLVING" QUE JE NE COMPRENDS PAS These are judgments given by subjects INDIVIDUAL'S JUDGMENTS PLUTOT QUE JUDGMENTS GIVEN BY INDIVIDUALS, NON ? after arguments in favor and against different possible stanceS have been considered BY WHOM? TO BE SPECIFIED HERE, I GUESS. Such judgments may differ from immediate judgments of desirability given in natural conditions NEED TO DEFINE IMMEDIATE AND NATURAL CONDITIONS, EVEN IN PASSING. Indeed, individuals may change opinion on the desirability of some course of action, and therefore revise their choice, while learning about the properties of objects, the consequences of some acts (or empirical knowledge useful to estimate the likelihood of some consequences), or the logical or empirical impossibilities INCOMPATIBILITIES? between consequences. This is particularly likely to happen when choosing among non everyday objects or acts whose consequences are not well known by the subject. TU NE CROIS PAS QU'UN EXEMPLE JOUET SERAIT UTILE ICI. MOI JE COMPRENDS, MAIS C'EST PARCE QUE JE SAIS DEJA CE QUE TU VEUX DIRE. UN INNOCENT COMPRENDRAIT-IL ? JE M'INTERROGE...

While numerous articles have discussed the concept of deliberation in the last decades IN RECENT DECADES, it remains unsettled UNCLEAR, to the best of my knowledge, how to observe a deliberated judgment in a systematic and reproducible fashion. There is a need to define this concept sufficiently precisely that, first, it is clear THERE IS A NEED TO DEFINE A CONCEPT WHICH IS SUFFICIENTLY UNEQUIVOCAL FOR IT TO BE CLEAR (i) WHAT COUNTS ETC what counts as a deliberated judgment (i.e., when “sufficient” \citep{meinard_justification_2020} or “correct” deliberation has happened); and second, it is clear which part of the claim is empirical and which part stems from an unfalsifiable conception of what “deliberated” means. ICI AUSSI, JE NE SAIS PAS SI QUELQU'UN QUI NE SAIT PAS DEJA CE QUE TU VEUX DIRE PEUT COMPRENDRE LE SECOND POINT. SUGGESTION: (ii) HOW DELIBERATED JUDGMENTS CAN BE EMPIRICALLY STUDIED

For conceptual clarity, I focus on judgments that take the form of a choice indicating a preference between two options. MORE SIMPLY: A CHOICE BETWEEN TWO OPTIONS?
This article TALK/INTERVENTION is therefore JE NE VOIS PAS LE LIEN LOGIQUE ENTRE CETTE PHRASE ET LA PRECEDENTE, QUI JUSTIFIE CE "THEREFORE" interested in defining the concept of \emph{deliberated choice} of a subject and in showing how to study it in an empirical manner. EMPIRICALLY
One goal of this endeavor is to ensure that disagreements about someone’s deliberated choice in a given context can in principle be solved by an empirical experiment, provided the very conception of what constitutes a deliberated choice is shared. CETTE PHRASE NE ME SEMBLE PAS CLAIRE DU TOUT. JE PENSE QUE LE LECTEUR SE DEMANDE "DISAGREEMENTS" ENTRE QUI ET QUI ? IL SEMBLE ICI ETRANGE DE PARLER DE DISAGREEMENTS COMME SI ON ETAIT DANS UNE DISCUSSION ALORS QUE, SI DESACCORD IL Y A ICI, C'EST ENTRE DES HYPOTHESES QUANT AU CONTENU EMPIRIQUE DES JUDGEMENTS DELIBERES DU COBAYE 

While the concept of deliberation in the deliberative democracy literature usually refers to multiple individuals debating, this article considers deliberation as related to a single individual: an individual deliberates (in the sense of: carefully weights) the stance she considers THE most appropriate while WHEN being confronted to arguments. In this sense, deliberated choices ARE come close to what \citet{rawls_theory_1999} calls reflective equilibrium (referring to an idea of \citet{goodman_fact_1983}), a concept that has much inspired the present proposal. C'EST PLUTOT DES JUGEMENTS BIEN PESES OBTENUS GRACE A LA PROCEDURE DE RECHERCHE DE L'EQUILIBRE REFLEXIF QUE LES JUGEMENTS DELIBERES SONT PROCHES, ME SEMBLE-T-IL

To define deliberated choices, one needs to start with the set of arguments under consideration, and the way the subject gets to know them, called hereafter the “exposure protocol”. WHAT IS THEREAFTER CALLED THAT WAY, NOT CLEAR TO ME
As a result, RESULT OF WHAT? JE DIRAIS PLUTPOT "ACCORDINGLY" we need always talk about the deliberated choice \emph{given some exposure protocol}. 
This accounts for the fact that the very way that PAS DE "THAT" the arguments are transmitted to an individual may affect her choice differently \citep{railton_facts_2003}. 
The notion of argument invoked here is an extremely large one, including texts, images, sounds, experiences (a hiking trip), and so on: anything that can be transmitted to a subject qualifies as a possible argument. In particular, such an “argument” need not have the logical structure of what is considered a proper argument in argumentation theory. This important feature of this proposal permits to avoid taking a position STANCE on what is a correct argument, and permits to study wide enough conceptions of deliberated choice, including those involving no paternalism \citep{cailloux_formal_2020}: anything that influences the choice of a subject, as judged by the subject herself and not by any external standard, may a priori be considered worth including in at least some conceptions of deliberated choice, thus including whatever some could consider as “incorrect reasoning” or “bad reasons”.

This approach yields a conceptual and practical clarification. It proposes a sharp distinction between the empirical content and the normative content of the concept of deliberated choices: the definition proposed here transfers the full normative weight on the choice of the exposure protocol; and leaves the rest of the disagreements to empirical settling. 
In supplement to a possibly helpful conceptual clarification, such a sharp separation may reveal practically useful for separation of concern as it permits to study deliberated choices empirically under various exposure protocols independently of possible disagreements about which norms are more appropriate for deliberation in which circumstances.

The exposure protocol defines a set of argument and, importantly, how arguments are sent to the subject and her subsequent choice observed. There is no explicit dependency on the internal process that occurs in the subject’s mind while she (possibly) processes the arguments, as it is not desirable to assume that the internal process could be observed. We consider subjects as black boxes and only require a capacity of observation of their choices after exposure to arguments.

The deliberated choice of a subject is defined as a function of the exposure protocol: someone’s deliberated choice equals the set of choices that resist every counter-arguments. The notion of resistance is itself defined according to the exposure protocol.

Studying choices resulting from exposure to arguments includes a difficulty: exposing an individual to arguments may change the individual’s stance, an effect that cannot be erased in order to submit the individual to an unrelated sequence of arguments. The observer thus cannot in general assume that multiple unrelated sequences of arguments can be tested on a given individual. This difficulty will be worked around by considering general theories of deliberated choices, that apply to sets of individuals. 

I will introduce properties of theories of deliberated choice, discuss how to falsify such theories, and present results about the settings in which it is guaranteed that suitable theories exist.

\hbadness=10000
\bibliography{simple}
\end{document}
