\RequirePackage[l2tabu, orthodox]{nag}
\documentclass[version=last, pagesize, twoside=off, bibliography=totoc, DIV=calc, fontsize=12pt, a4paper, french, english]{scrartcl}
\input{preamble/packages}
\input{preamble/math_basics}
\input{preamble/math_mine}
\input{preamble/redac}
\usepackage[normalem]{ulem}
%\input{preamble/draw}
%\input{preamble/acronyms}
\addtokomafont{labelinglabel}{\sffamily\bfseries}
\DeclareMathAlphabet{\mathup}{OT1}{\familydefault}{m}{n}

%I find these settings useful in draft mode. Should be removed for final versions.
	%Which line breaks are chosen: accept worse lines, therefore reducing risk of overfull lines. Default = 200.
		\tolerance=2000
	%Accept overfull hbox up to...
		\hfuzz=2cm
	%Reduces verbosity about the bad line breaks.
		\hbadness 5000
	%Reduces verbosity about the underful vboxes.
		\vbadness=1300

\begin{document}
\title{Defining deliberated choice and theories thereof}
\author{Olivier Cailloux}
\affil{Université Paris-Dauphine, PSL Research University, CNRS, LAMSADE, 75016 PARIS, FRANCE\\
  \href{mailto:olivier.cailloux@dauphine.fr}{olivier.cailloux@dauphine.fr}
}
\date{}
\makeatletter
\hypersetup{
  pdfsubject={Epistemology},
  pdfkeywords={Decision support, Decision making, Argumentation}
}
\makeatother
\maketitle

Studying subjects perform acts of choice is the basis of much work in economics.
Most of these approaches consider only acts of choice that are performed without exposure to arguments and counter-arguments. Such acts will be named here “shallow” choice acts. 
Examples include studying willingness to pay, or studying influence on individual choices of varying frames as in the celebrated work of Kahneman and Tversky \citep{bell_descriptive_1988, kahneman_thinking_2012}.

On the other hand, since at least the seminal work of \citet{fishkin_when_2011}, another sort of acts of choice is emerging as also worth studying.
These are acts performed by individuals 
after having considered arguments in favor and against different possible stances. 
Such acts may differ from shallow choice acts. 
Indeed, individuals may change opinion on the desirability of some course of action, and therefore revise their choice, while learning about the properties of objects, the consequences of some acts (or empirical knowledge useful to estimate the likelihood of some consequences), or the logical or empirical incompatibilities between consequences. 
For example, an individual may spontaneously pick a milkshake rather than a can of coke when choosing what to drink but, upon learning their respective calorific content, rather opt for the can of coke.
Discrepancies between shallow and deliberated choices are particularly likely to happen when choosing among non everyday objects or acts whose consequences are not well known by the subject. 

While numerous articles have discussed the concept of deliberation in recent decades, it remains unclear, to the best of my knowledge, how to observe a deliberated judgment in a systematic and reproducible fashion. 
There is a need to define a concept which is sufficiently unequivocal for it to be clear 
(i) what counts as a deliberated judgment (i.e., when “sufficient” \citep{meinard_justification_2020} or “correct” deliberation has happened); 
and (ii) how deliberated judgments can be empirically studied.

For conceptual clarity, I focus on judgments that take the form of a choice between two options.

This talk is interested in defining the concept of \emph{deliberated choice} of a subject and in showing how to study it empirically.
One goal of this endeavor is to ensure that individual’s deliberated choices in a given context can in principle be established empirically, provided the very conception of what constitutes a deliberated choice is shared.

While the concept of deliberation in the deliberative democracy literature usually refers to multiple individuals debating, this article considers deliberation as related to a single individual: an individual deliberates (in the sense of: carefully weights) the stance that she considers the most appropriate when being confronted to arguments. In this sense, deliberated choices are related to judgments obtained by a procedure that \citet{rawls_theory_1999} calls the search for a reflective equilibrium (referring to an idea of \citet{goodman_fact_1983}), a concept that has much inspired the present proposal.

To define deliberated choices, one needs to start with the set of arguments under consideration, and the procedure by which the subject gets to know them. That procedure is called hereafter the “exposure protocol”.
Accordingly, we need always talk about the deliberated choice \emph{given some exposure protocol}. 
This accounts for the fact that the very way the arguments are transmitted to an individual may affect her choice differently \citep{railton_facts_2003}. 
The notion of argument invoked here is an extremely large one, including texts, images, sounds, experiences (a hiking trip), and so on: anything that can be transmitted to a subject qualifies as a possible argument. In particular, such an “argument” need not have the logical structure of what is considered a proper argument in argumentation theory. This important feature of this proposal permits to avoid taking a stance on what is a correct argument, and permits to study wide enough conceptions of deliberated choice, including those involving no paternalism \citep{cailloux_formal_2020}: anything that influences the choice of a subject, as judged by the subject herself and not by any external standard, may a priori be considered worth including in at least some conceptions of deliberated choice, thus including whatever some could consider as “incorrect reasoning” or “bad reasons”.

This approach yields a conceptual and practical clarification. It proposes a sharp distinction between the empirical content and the normative content of the concept of deliberated choices: the definition proposed here transfers the full normative weight on the choice of the exposure protocol; and leaves the rest of the disagreements to empirical settling. 
In supplement to a possibly helpful conceptual clarification, such a sharp separation may reveal practically useful for separation of concern as it permits to study deliberated choices empirically under various exposure protocols independently of possible disagreements about which norms are more appropriate for deliberation in which circumstances.

The exposure protocol defines a set of argument and, importantly, how arguments are sent to the subject and her subsequent choice observed. There is no explicit dependency on the internal process that occurs in the subject’s mind while she (possibly) processes the arguments, as it is not desirable to assume that the internal process could be observed. We consider subjects as black boxes and only require a capacity of observation of their choices after exposure to arguments.

The deliberated choice of a subject is defined as a function of the exposure protocol: someone’s deliberated choice equals the set of choices that resist every counter-arguments. The notion of resistance is itself defined according to the exposure protocol.

Studying choices resulting from exposure to arguments includes a difficulty: exposing an individual to arguments may change the individual’s stance, an effect that cannot be erased in order to submit the individual to an unrelated sequence of arguments. The observer thus cannot in general assume that multiple unrelated sequences of arguments can be tested on a given individual. This difficulty will be worked around by considering general theories of deliberated choices, that apply to sets of individuals. 

I will introduce properties of theories of deliberated choice, discuss how to falsify such theories, and present results about the settings in which it is guaranteed that suitable theories exist.

\hbadness=10000
\bibliography{simple}
\end{document}
